% 		HYPERREF
\usepackage[
	pdfauthor={Der Autor des Dokuments},
	pdftitle={Der Titel Ihrer Arbeit},
	hidelinks=true % keine roten Markierungen bei Links
]{hyperref}

%		FONT AND INPUT ENCODING 
%
\usepackage[T1]{fontenc}
\usepackage[utf8]{inputenc}

%
%		CALCULATIONS 
\usepackage{calc} % wird benötigt, um den kleinen Abstand nach der footsepline zu erzeugen

%		LANGUAGE SETTINGS
%
\usepackage[ngerman]{babel} 	% Voreinstellung Deutsch 
\usepackage[german=quotes]{csquotes} 	% Richtiges Setzen der Anführungszeichen mit \enquote{}  

%\usepackage[english]{babel}   %Für englische Arbeiten
%\usepackage{csquotes} 	% Richtiges Setzen der Anführungszeichen mit \enquote{}  


%		BIBLIOGRAPHY SETTINGS
%
\usepackage[backend=biber, autocite=footnote, style=authoryear-icomp, dashed=false]{biblatex} 	%Define Bibliography style 
\DefineBibliographyStrings{ngerman}{  %Change u.a. to et al. (german only!)
	andothers = {{et\,al\adddot}},             
} 
\setlength{\bibparsep}{\parskip}		%add some space between biblatex entries in the bibliography
\addbibresource{bibliography.bib}	%Add file bibliography.bib as biblatex resource 


%		GLOSSARIES AND ACRONYMS
\usepackage[printonlyused]{acronym}


%		LISTINGS
\usepackage{listings}	%Format Listings properly 
\renewcommand{\lstlistlistingname}{Quelltextverzeichnis}
\lstset{numbers=left,
	numberstyle=\tiny,
	captionpos=b,
	basicstyle=\ttfamily\small}


%		EXTRA PACKAGES 
\usepackage{lipsum}    %Blindtext
\usepackage{graphicx} %Verschiedene Grafikformate einbinden  
\usepackage[german]{varioref} 	%Bezüge etwas hübscher herstellen mit \vref
\usepackage{caption}	%bessere Captions 
\usepackage{booktabs} %schönere Tabellen


%		Algorithmen
\usepackage{algorithm}  
\usepackage{algpseudocode}
\renewcommand{\listalgorithmname}{Algorithmenverzeichnis }
\floatname{algorithm}{Algorithmus}


%		FONT SELECTION: Entweder Latin Modern oder Times / Helvetica
\usepackage{lmodern} %Latin modern font 
%\usepackage{mathptmx}  %Helvetica / Times New Roman fonts (2 lines)
%\usepackage[scaled=.92]{helvet} %Helvetica / Times New Roman fonts (2 lines)

%		PAGE HEADER / FOOTER 
%	    Achtung: In einzelnen Abschnitten der master.tex - Datei wird ggf. der innere Header redefiniert! 
\RequirePackage[automark,headsepline,footsepline]{scrpage2}
\pagestyle{scrheadings}
\renewcommand*{\pnumfont}{\upshape\sffamily}
\renewcommand*{\headfont}{\upshape\sffamily}
\renewcommand*{\footfont}{\upshape\sffamily}
\renewcommand{\chaptermarkformat}{}

\clearscrheadfoot

\ifoot[\rule{0pt}{\ht\strutbox+\dp\strutbox}DHBW Mannheim]{\rule{0pt}{\ht\strutbox+\dp\strutbox}DHBW Mannheim}
\ofoot[\rule{0pt}{\ht\strutbox+\dp\strutbox}\pagemark]{\rule{0pt}{\ht\strutbox+\dp\strutbox}\pagemark}

\ohead{\headmark}





