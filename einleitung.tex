\chapter{Einleitung}
Vorab: Diese Vorlage basiert auf modernen \LaTeX--Paketen, u.\,A. auf Biblatex, Komascript, Glossaries, Booktabs, etc. 
Sie benötigen daher eine aktuelle Installation. Kommentieren Sie die nicht benötigten Verzeichnisse (z.\,B. \texttt{listoflistings}) aus und passen Sie die Titelseite 
entsprechend an. 

Mit dem Befehl \texttt{autocite} kann zitiert werden, z.\,B. so\autocite[Vgl.][S. 18ff.]{Tanenbaum2012}.
Ein Akronym kann man mit dem Befehl \texttt{ac} verwenden, z.\,B. so: \ac{DHBW}. Im weiteren Verlauf wird das 
Acronym dann nur noch in der Kurzform dargestellt: \ac{DHBW}. Die Übernahme zitierter Quellen oder Akronyme in die 
entsprechenden Verzeichnisse erfolgt automatisch. Das hier ist in \enquote{Text in Anführungszeichen}.

Und das hier ist ein Absatz.



\section{Unterabschnitt}
\lipsum[1]

\subsection{UnterUnterAbschnitt mit Tabelle}
In Tabelle \vref{tab:tabelle1} ist eine Tabelle abgebildet, die mit dem Befehl \texttt{vref} referenziert wurde. Gleiches kann man auch mit Abbildungen 
machen, wie z.\,B. mit der Abbildung \vref{fig:test}. \LaTeX~ kümmert sich darum, wo die Abbildungen gesetzt werden und passt den Text entsprechend an.
In der Tabelle sind die Spalten mit einer Fixlänge von 3cm, zentriert, rechts- und linksbündig definiert. Abbildungen sollten -- falls möglich -- als Vektor-PDF eingebunden 
werden, da die Abbildungen dann beliebig skalieren können.


\lipsum[1-2]
\begin{table}
	\centering
	\begin{tabular}{p{3cm}crl}
		\textbf{Spalte 1} & \textbf{Spalte 2} & \textbf{Spalte 3} & \textbf{Spalte 4}\\\toprule
		Zeile 1 Spalte 1 &  Zeile 1 Spalte 2 & Zeile 1 Spalte 3 & Zeile 1 Spalte 4\\
		Zeile 2 Spalte 2 &  Zeile 2 Spalte 2 & Zeile 2 Spalte 3 & Zeile 2 Spalte 4\\\midrule
		Zeile 3 Spalte 1 &  Zeile 3 Spalte 2 & Zeile 3 Spalte 3 & Zeile 3 Spalte 4\\
		Zeile 4 Spalte 1 &  Zeile 4 Spalte 2 & Zeile 4 Spalte 3 & Zeile 4 Spalte 4\\\bottomrule
	\end{tabular}
	\caption[Testtabelle]{\label{tab:tabelle1}Testtabelle}
\end{table}
\lipsum[1-2]

\begin{figure}
	\centering 
	\includegraphics{img/firmenlogo.jpg}
	\captionsetup{format=hang}
	\caption[Optionaler Kurztitel für das Abbildunggsverzeichnis]{\label{fig:test}Demo-Abbildung, um zu verdeutlichen, wie gleitende Objekte gesetzt werden und wie entsprechend die Quelle zitiert wird. \\Quelle: \cite[][S. 223]{Lamport1978}}
\end{figure}
	
Das Einbinden eines Listings mit der entsprechenden Umgebung ist auch kein Problem, wie man in Listing \vref{lst:helloworld} sehen kann. Schauen Sie sich hierzu das \texttt{listings}-Paket an! 
	
	
\lstset{language=Java}
\begin{lstlisting}[caption={Hello World!}, label={lst:helloworld}]
public static void main(String args[]) {
   System.out.println("Hello World!");
}
\end{lstlisting}

Auch mathematische Ausdrücke können mit \LaTeX~ sehr gut gesetzt werden, wie man anhand der Gleichungen \vref{eqn:e1} und \vref{eqn:e2} sehen kann. -- konsultieren Sie hierzu bitte entsprechende Dokumentationen, die Online zur Verfügung stehen.
\begin{equation}
\left|{1\over N}\sum_{n=1}^N \gamma(u_n)-{1\over 2\pi}\int_0^{2\pi}\gamma(t){\rm d}t\right| \le {\varepsilon\over 3}.\\
\label{eqn:e1}
\end{equation}

\begin{equation}
f(x)=x^2
\label{eqn:e2}
\end{equation}


