%
%   Prof. Dr. Julian Reichwald  
%   auf Basis einer Vorlage von Prof. Dr. Jörg Baumgart
%   DHBW Mannheim 
%  
%
%	ACHTUNG: Für das Erstellen des Literaturverzeichnisses wird das modernere Paket biblatex
%			 in Kombination mit biber verwendet -- nicht mehr das ältere BibTex!  
% 			 Bitte stellen Sie ggf. Ihre TeX-Umgebung 
% 			 entsprechend ein (z.B. TeXStudio: Einstellungen --> Erzeugen --> Standard Bibliographieprogramm: biber)
%


\documentclass[
	12pt,
	BCOR=10mm,1
	headinclude=on,
	footinclude=off,
	parskip=half,
	bibliography=totoc,
	listof=entryprefix,
	toc=listof, 
	pointlessnumbers,
	plainfootsepline]{scrreprt}

%	Konfigurationsdatei einziehen
% 		HYPERREF
\usepackage[
	pdfauthor={Der Autor des Dokuments},
	pdftitle={Der Titel Ihrer Arbeit},
	hidelinks=true % do not paint ugly red boxes around links in pdf
]{hyperref}

%		FONT AND INPUT ENCODING 
%
\usepackage[T1]{fontenc}
\usepackage[utf8]{inputenc}


%		LANGUAGE SETTINGS
%
\usepackage[ngerman]{babel} 	%German Language 
\usepackage[german=quotes]{csquotes} 	%Enable \enquote{} Statement for language-specific quotes 
%\usepackage[ngerman]{english}   %Für englische Arbeiten


%		BIBLIOGRAPHY SETTINGS
%
\usepackage[backend=biber, autocite=footnote, style=authoryear-icomp, dashed=false]{biblatex} 	%Define Bibliography style 
\DefineBibliographyStrings{ngerman}{  %Change u.a. to et al. (german only!)
	andothers = {{et\,al\adddot}},             
} 
\setlength{\bibparsep}{\parskip}		%add some space between biblatex entries in the bibliography
\addbibresource{bibliography.bib}	%Add file bibliography.bib as biblatex resource 


%		GLOSSARIES AND ACRONYMS
\usepackage[printonlyused]{acronym}


%		LISTINGS
\usepackage{listings}	%Format Listings properly 
\renewcommand{\lstlistlistingname}{Quelltextverzeichnis}
\lstset{numbers=left,
	numberstyle=\tiny,
	captionpos=b,
	basicstyle=\ttfamily\small}


%		EXTRA PACKAGES 
\usepackage{lipsum}    %Blind-Text Generation
\usepackage{graphicx} %Include graphics 
\usepackage[german]{varioref} 	%Cross References made easy with \vref
\usepackage{caption}	%Better Captions 
\usepackage{booktabs} %nicer tables 


%		Algorithmen
\usepackage{algorithm}
\usepackage{algpseudocode}
\renewcommand{\listalgorithmname}{Algorithmenverzeichnis }
\floatname{algorithm}{Algorithmus}


%		FONT SELECTION: Either use latin modern or Times / Helvetica combination
\usepackage{lmodern} %Latin modern font 
%\usepackage{mathptmx}  %Helvetica / Times New Roman fonts (2 lines)
%\usepackage[scaled=.92]{helvet} %Helvetica / Times New Roman fonts (2 lines)

%		PAGE HEADER / FOOTER 
\RequirePackage[automark,headsepline,footsepline]{scrpage2}
\pagestyle{scrheadings}
\renewcommand*{\pnumfont}{\upshape\sffamily}
\renewcommand*{\headfont}{\upshape\sffamily}
\renewcommand*{\footfont}{\upshape\sffamily}
\renewcommand{\chaptermarkformat}{}

\clearscrheadfoot

\ifoot[DHBW Mannheim]{DHBW Mannheim}
\ofoot[\pagemark]{\pagemark}

\ihead{\chaptername~\thechapter}
\ohead{\headmark}







\begin{document}

\begin{titlepage}
\begin{minipage}{\textwidth}
		\vspace{-2cm}
		\noindent \includegraphics[scale=0.71]{img/firmenlogo.jpg} \hfill   \includegraphics{img/logo.jpg}
\end{minipage}
\vspace{1em}
\sffamily
\begin{center}
	\textsf{\large{}Duale Hochschule Baden-W\"urttemberg\\[1.5mm] Mannheim}\\[2em]
	\textsf{\textbf{\Large{}Bachelorarbeit}}\\[3mm]
	\textsf{\textbf{\DerTitelDerArbeit}} \\[1.5cm]
	\textsf{\textbf{\Large{}Studiengang Wirtschaftsinformatik}\\[3mm] \textsf{Studienrichtung Software Engineering}}
	
	\vspace{3em}
	\textsf{\Large{Sperrvermerk}}
\vfill

\begin{minipage}{\textwidth}

\begin{tabbing}
	Wissenschaftlicher Betreuer: \hspace{0.85cm}\=\kill
	Verfasserin: \> \DerAutorDerArbeit \\[1.5mm]
	Matrikelnummer: \> 123456 \\[1.5mm]
	Firma: \> Musterfirma 123 \\[1.5mm]
	Abteilung: \> Softwareentwicklung \\[1.5mm]
	Kurs: \> WWI 99 SEZ \\[1.5mm]
	Studiengangsleiter: \> Prof. Dr. Julian Reichwald  \\[1.5mm]
	Wissenschaftlicher Betreuer: \> Dr. Max Mustermann \\
	\> test@test.com \\
	\> +49 151 / 123 456 \\[1.5mm]
	Firmenbetreuer: \> Moritz Testname \\
	\> test@test.com \\
	\> +49 151 / 123 456 \\[1.5mm]
	Bearbeitungszeitraum: \> 01.01.1970 -- 31.12.2099
\end{tabbing}
\end{minipage}

\end{center}

\end{titlepage}

\pagenumbering{roman} % Römische Seitennummerierung
\normalfont

%--------------------------------
% Verzeichnisse - nicht benötige Verzeichnisse bitte auskommentieren / löschen.
%--------------------------------

%	Inhaltsverzeichnis 
\tableofcontents

%	Abbildungsverzeichnis 
\listoffigures

%	Tabellenverzeichnis 
\listoftables

%	Listingsverzeichnis 
 \lstlistoflistings 

% 	Algorithmenverzeichnis
\listofalgorithms

% 	Abkürzungsverzeichnis (siehe Datei acronyms.tex!)
\clearpage
\chapter*{Abkürzungsverzeichnis}	
\addcontentsline{toc}{chapter}{Abkürzungsverzeichnis}


\begin{acronym}[RDBMS]
	\acro{DHBW}{Duale Hochschule Baden-Württemberg}
	\acro{RDBMS}{Relational Database Management System}
	\acro{BMBF}{Bundesministerium für Bildung und Forschung}	
\end{acronym}


\enquote{test}
\newpage
\pagenumbering{arabic}  % Arabische Seitenzahlen

%	Einleitungs-Datei einziehen
\chapter{Gebrauchsanleitung}
Diese Vorlage verwendet u.\,a. die folgenden Pakete und Programme: 
\begin{itemize}
	\item\texttt{KOMA-Script} und der Dokumentenklasse \texttt{scrreprt}
	\item \texttt{babel} für die Spracheinstellungen
	\item \texttt{csquotes} für sprachabhängige Anführungszeichen 
	\item \texttt{acronym} für das Erstellen des Abkürzungsverzeichnisses 
	\item \texttt{booktabs} für das typografisch korrekte Setzen von Tabellen 
	\item \texttt{varioref} für einfaches Querreferenzieren 
	\item \texttt{listings} für schöne Quelltexte
	\item \texttt{algorithmicx} für schöne Algorithmen
	\item \texttt{bibltatex} und \texttt{biber} für die Erstellung des Literaturverzeichnisses 
\end{itemize}
Alle Konfigurationen dieser Vorlage können in der Datei \texttt{config.tex} eingesehen und ggf. verändert werden. Bitte schauen Sie sich die entsprechenden Dokumentationen 
der Pakete an (\url{https://www.ctan.org}), um deren Verwendung und Möglichkeiten jenseits der hier gezeigten Beispiele zu verstehen.

Die Datei \texttt{master.tex} ist die Hauptdatei, von der aus die anderen Dateien eingezogen werden. Die weiteren Dateien sind in Tabelle \vref{tab:dateien} beschrieben.
 
\begin{table}
	\centering
\begin{tabular}{lp{10cm}}
	\textbf{Dateiname} & \textbf{Beschreibung}\\\toprule
	\texttt{config.tex} & Konfigurationseinstellungen der einzelnen Pakete\\
	\texttt{acronyms.tex} & Definition von Abkürzungen. Fügen Sie die von Ihnen verwendeten Abkürzungen in diese Datei ein.\\
	\texttt{titlepage.tex} & Titelseite der Arbeit. Passen Sie die Titelseite mit den entsprechenden Daten an.\\
	\texttt{anleitung.tex} & Diese Anleitung\\ 
	\texttt{bibliography.bib}&  Die Literaturdatenbank -- hier können SIe verwendete Literatur einpflegen.\\
	\texttt{appendix.tex} & Anhang bzw. Anhänge \\\bottomrule
\end{tabular}
\caption{\label{tab:dateien}Übersicht über die Dateien der Vorlage}
\end{table}

\section{Übersetzung von \LaTeX-Dateien}
Die Übersetzung von \LaTeX-Dateien erfolgt in mehreren Schritten und unter der Zuhilfenahme unterschiedlicher Programme. Zum Übersetzen dieser Vorlage sind die Programme.

Das Abkürzungsverzeichnis mittels des \texttt{glossaries}-Paketes muss mittels \texttt{makeindex} erstellt werden. Sollten ggf. Abkürzungen nicht im Verzeichnis enthalten sein,
bitte löschen Sie die bei der Übersetzung des Dokumentes anfallenden temporären Dateien und führen Sie \texttt{pdflatex}, \texttt{biber} (für das Erstellen des Literaturverzeichnisses) sowie \texttt{makeindex} bzw. \texttt{makeglossaries} erneut aus.

Mit dem Befehl \texttt{autocite} kann zitiert werden, z.\,B. so\autocite[Vgl.][S. 18ff.]{ME12}.
Ein Akronym kann man mit dem Befehl \texttt{ac} verwenden, z.\,B. so: \ac{DHBW}. Im weiteren Verlauf wird das 
Acronym dann nur noch in der Kurzform dargestellt: \ac{DHBW}. Die Übernahme zitierter Quellen oder Akronyme in die 
entsprechenden Verzeichnisse erfolgt automatisch. 

Soll ein Text in Anführungszeichen gesetzt werden, kann dies  \enquote{so erreicht werden}. Die Anführungszeichen ändern sich automatisch auf die 
jeweiligen Länderspezifika, wenn die Spracheinstellung des \texttt{babel}-Pakets geändert wird.




\section{Beispiele}
\lipsum[1]

\subsection{Unterabschnitte}
Es gibt neben \texttt{chapter} auch noch  \texttt{section}, \texttt{subsection}, \texttt{subsubsection} etc. Eine zu starke Untergliederung des Textes sollte jedoch vermieden werden (z.\,B. ein Abschnitt 3.4.2.5.3). 

\subsection{Tabellen und Abbildungen}
Tabellen und Abbildungen sind sogenannte \textit{Floating Objects}, d.\,h. \LaTeX\ setzt diese Objekte an Positionen, die satztechnisch geeignet sind. Daher kann es vorkommen, dass Tabellen oder Abbildungen auf einer anderen Seite erscheinen, die dann referenziert werden müssen. Hier ein Beispiel dafür: 

In Tabelle \vref{tab:tabelle1} ist eine Tabelle abgebildet, die mit dem Befehl \texttt{vref} referenziert wurde. Gleiches kann man auch mit Abbildungen 
machen, wie z.\,B. mit der Abbildung \vref{fig:test}. \LaTeX~ kümmert sich darum, wo die Abbildungen gesetzt werden und passt den Text der Referenz entsprechend an. Soll nur die Nummerierung in den Text geschrieben werden, dann kann auch der Befehl \texttt{ref} verwendet werden.
Abbildungen sollten -- falls möglich -- als Vektor-PDF eingebunden 
werden, da die diese dann beliebig skalieren können.

\lipsum[1]
\begin{table}
	\centering
	\begin{tabular}{p{3cm}crl}
		\textbf{Spalte 1} & \textbf{Spalte 2} & \textbf{Spalte 3} & \textbf{Spalte 4}\\\toprule
		Zeile 1 Spalte 1 &  Zeile 1 Spalte 2 & Zeile 1 Spalte 3 & Zeile 1 Spalte 4\\
		Zeile 2 Spalte 2 &  Zeile 2 Spalte 2 & Zeile 2 Spalte 3 & Zeile 2 Spalte 4\\\midrule
		Zeile 3 Spalte 1 &  Zeile 3 Spalte 2 & Zeile 3 Spalte 3 & Zeile 3 Spalte 4\\
		Zeile 4 Spalte 1 &  Zeile 4 Spalte 2 & Zeile 4 Spalte 3 & Zeile 4 Spalte 4\\\bottomrule
	\end{tabular}
	\caption[Testtabelle]{\label{tab:tabelle1}Testtabelle}
\end{table}
\lipsum[1-2]

\begin{figure}
	\centering 
	\includegraphics{img/firmenlogo.jpg}
	\captionsetup{format=hang}
	\caption[Optionaler Kurztitel für das Abbildunggsverzeichnis]{\label{fig:test}Demo-Abbildung, um zu verdeutlichen, wie gleitende Objekte gesetzt werden und wie entsprechend die Quelle zitiert wird. \\Quelle: \cite[][S. 223]{TD15}}
\end{figure}
	
\subsection{Listings}	

Das Einbinden eines Listings mit der entsprechenden Umgebung ist auch kein Problem, wie man in Listing \vref{lst:helloworld} sehen kann. Schauen Sie sich hierzu das \texttt{listings}-Paket an! 
		
		\newpage
		
		
\lstset{language=Java}
\begin{lstlisting}[caption={Hello World!}, label={lst:helloworld}]
public static void main(String args[]) {
   System.out.println("Hello World!");
}
\end{lstlisting}


\subsection{Mathematische Formeln}
Auch mathematische Ausdrücke können mit \LaTeX~ sehr gut gesetzt werden, wie man anhand der Gleichungen \vref{eqn:e1} und \vref{eqn:e2} sehen kann -- konsultieren Sie hierzu bitte entsprechende Dokumentationen, die Online zur Verfügung stehen.
\begin{equation}
\left|{1\over N}\sum_{n=1}^N \gamma(u_n)-{1\over 2\pi}\int_0^{2\pi}\gamma(t){\rm d}t\right| \le {\varepsilon\over 3}.\\
\label{eqn:e1}
\end{equation}

\begin{equation}
f(x)=x^2
\label{eqn:e2}
\end{equation}


\subsection{Algorithmen}
Algorithmen können als Pseudocodes dargestellt und referenziert werden, wie z.\,B. in Algorithmus \vref{alg:euclid} -- sogar bis auf Zeilennummern
(siehe die \texttt{while}-Anweisung in Zeile \vref{alg:euclid:while}). Schauen Sie sich hierzu bitte das Paket \texttt{algorithmicx} an.



\begin{algorithm}
\begin{algorithmic}[1]
\Procedure{Euclid}{$a,b$}\Comment{The g.c.d. of a and b}
   \State $r\gets a\bmod b$
   \While{$r\not=0$}\Comment{We have the answer if r is 0} \label{alg:euclid:while}
      \State $a\gets b$
      \State $b\gets r$
      \State $r\gets a\bmod b$
   \EndWhile\label{euclidendwhile}
   \State \textbf{return} $b$\Comment{The gcd is b}
\EndProcedure
\end{algorithmic}
\caption{Euklid'scher Algorithmus}\label{alg:euclid}
\end{algorithm}



%	Literaturverzeichnis 	
\printbibliography	

% Der Anhang beginnt hier - jedes Kapitel wird alphabetisch aufgezählt. (Anhang A, B usw.)
\appendix

\chapter{Testanhang}

\section{Subtestanhang}

\chapter{Noch ein Testanhang}



\end{document}